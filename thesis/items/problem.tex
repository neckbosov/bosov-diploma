% !TEX root = ../thesis-main.tex
\specialsection{Постановка задачи}
\textbf{Целью} данной выпускной работы является
создать расширяемый генератор случайных программ для учебных задач, используемых в курсах по обучению языкам программирования.

Основные задачи которые необходимо сделать:
\begin{enumerate}[label=\alph*.]
    \item Изучить существующие системы генерации случайных программ на предмет возможности их
          настройки и применимости результатов их работы в учебных целях.
    \item Создать систему
          генерации программ с возможностью настройки параметров для одного языка программирования (Python)
    \item Адаптировать систему к возможности поддержки других языков программирования.
\end{enumerate}

\textbf{Объектом} моего исследования являются инструменты генерации программного кода, а
\textbf{предметом} исследования --- применимость инструментов генерации кода для создания учебных задач.

Данная работа является развитием идеи, изложенной в статье \cite{haphiz}, в сторону расширяемости и поддержки разных языков программирования.