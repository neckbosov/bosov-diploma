\section{Разработка инструмента генерации программ}
Сравнительный анализ существующих решений для генерации программ показал, что
инструменты, используемые на практике, не подходят для учебных целей. Поэтому было принято решение
разработать собственную систему генерации программ для учебных задач.

\subsection{Требования к системе генерации}
Разрабатываемый инструмент должен обладать следующими возможностями:
\begin{itemize}
    \item Возможность генерации базовых элементов языка программирования;
    \item Поддержка генерации кода на разных языках, чтобы иметь возможность использовать данный
          инструмент в разных обучающих курсах;
    \item Расширяемость, что включает в себя:
          \begin{itemize}
              \item Гибкую систему создания шаблонов для генерации задач;
              \item Возможность настройки параметров генерации;
              \item Высокую вариативность задач, поддержку рандомизации отдельных элементов кода
          \end{itemize}
\end{itemize}

\subsection{Шаблоны программ}
\subsubsection{Общие единицы кода для разных языков}
\subsubsection{DSL}
\subsection{Примеры задач}
