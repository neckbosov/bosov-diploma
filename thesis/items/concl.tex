% !TEX root = ../thesis-main.tex
\specialsection{Заключение}

Целью и основной задачей дипломной работы было создание генератора программ для обучения
программированию. В ходе выполнения работы были получены следующие результаты:

\begin{enumerate}
    \item Было проведено исследование существующих решений по генерации программ на
          предмет их применимости в обучающих целях. В ходе исследования были выявлены достоинства и
          недостатки имеющихся решений.
    \item Были сформулированы требования к генератору обучающих программ, учитывающие используемые
          модели в аналогах, их достоинства и недостатки. Из основных недостатков были выявлены: отсутствие возможности
          задать основную логику программы; малофункциональные модели задания шаблона кода при их
          наличии; отсутствие веб-интерфейса; отсутствие возможности хранения шаблонов и
          сгенерированных программ; отсутствие возможности генерации изображений.
    \item Создан предметно-ориентированный язык для описания задач (шаблонов кода), на основе которых генерируются примеры программ.
    \item Был реализован инструмент генерации программ, поддерживающий управление
          шаблонами кода, имеющий потенциал к расширению в сторону поддержки новых языков
          программирования, в настоящий момент поддерживающий генерацию кода на языке Python
          с помощью основных синтаксических конструкций.
\end{enumerate}

Таким образом, цель данной работы достигнута в полном объеме.