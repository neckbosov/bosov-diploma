\section{Реализация инструмента генерации программ}

\subsection{Используемые технологии}

Для разработки инструмента был выбран язык программирования Kotlin \cite{kotlin}. Это язык
программирования, разработанный компанией JetBrains в 2010 году. Основными
преимуществами этого языка программирования являются:
\begin{itemize}
    \item Кроссплатформенность
    \item Возможность интеграции с различными популярными системами сборки
          (Maven, Gradle, etc.)
    \item Возможность интеграции с java без переписывания имеющегося кода
    \item Возможность создания DSL.
\end{itemize}

Для автоматизации развертывания системы генерации, поддерживания окружения для хранилища и
системы проверки используются технологии Docker \cite{docker} и docker-compose \cite{docker-compose}.
Благодаря Docker можно создать изолированное окружение (контейнер) для компонента, а docker-compose
позволяет объединять контейнеры в единую локальную сеть. В данном проекте созданы отдельные
контейнеры для базы данных и веб-сервера.


\subsubsection{Менеджер шаблонов}
Для хранения и пересылки шаблонов была добавлена поддержка конвертации шаблона в формат JSON~---
один из самых распространенных форматов обмена данных, который также может использоваться и для
хранения информации (см \ref{db}). Из-за сложной структуры классов,
в виде экземпляров которых представляется шаблон (\ref{dsl-classes}), для сериализации шаблонов в
JSON была выбрана библиотека \texttt{Kotlinx.serialization} \cite{kotlinx-serialization},
которая поддерживает сериализацию объектов языка программирования Kotlin и умеет работать
с наследованием и полиморфизмом.

Для отправки запросов на создание и удаление через Интернет был использован
Http~-клиент фреймворка Ktor \cite{ktor}, так как данный клинт поддерживает
сериализацию тела запроса (и ответа) с помощью \texttt{Kotlinx.serialization}.

\subsubsection{Веб-сервер}
Для реализации веб-сервера в связи с использованием для сериализации \texttt{Kotlinx.serialization}
было принято решение использовать Http~-сервер фреймворка Ktor \cite{ktor}, который поддерживает
данный способ сериализации запросов, а также имеет простой и элегантный API и возможность расширения
с помощью плагинов.

\subsubsection{База данных}
\label{db}
Так как шаблоны программ конвертируются и передаются в формате JSON, то было принято решение
переиспользовать имеющуюся логику и хранить их также в этом формате. В данном проекте используется
\texttt{MongoDB}~\cite{mongodb}~--- база данных, которая работает именно с форматом JSON.

Для работы с \texttt{MongoDB} из серверного кода была выбрана библиотека \texttt{kmongo}. Данная
библиотека является самой популярной библиотекой для работы с \texttt{MongoDB} из языка
\texttt{Kotlin}, также она поддерживает конвертацию в JSON и обратно с помощью
\texttt{Kotlinx.serialization}, что также упрощает работу с шаблонами.

Поля, которые хранятся в соответствующих коллекциях, описаны в \ref{db-model}. Однако в
данной библиотеке отсутствует поддержка хранение сырых байтов в базе данных, поэтому для
хранения изображения кода в базе байты изображения, с помощью алгоритма Base64\cite{rfc4648},
преобразуются в строку и в таком виде сохраняются в базу.

\subsubsection{Исполнитель программ}
Для компиляции (при необходимости), выполнения, форматирования кода и генерации изображений используются
отдельные программы, запускающиеся во внешнем окружении. По соображениям безопасности и для
ограничения нагрузки на инфраструктуру, на которой работает сервер, одновременно может работать
только один исполнитель, для этого в веб-сервере код, выполняющий программы во внешнем окружении,
запускается под блокировкой мьютекса. Для файлов, создаваемых во время выполнения, выделен
отдельный каталог.

Для генерации изображений по коду используется библиотека\\
\texttt{python-pygments}\cite{pygments} и поставляемая
с ней программа \texttt{pygmentize}. Данная библиотека написана на языке программирования
Python\cite{python} и поддерживает большинство популярных языков
программирования\cite{pygments-languages}. Изображения генерируются в формате png.

Для остальных задач из списка используются утилиты, специфичные для конкретного языка программирования.
Пример таких утилит для языка программирования Python\cite{python}:

\begin{itemize}
    \item Утилита для форматирования: \texttt{autopep8}\cite{autopep8}
    \item Компилятор: компиляция не требуется
    \item Утилита для выполнения кода: стандартный интерпретатор \texttt{python} версии 3.10
\end{itemize}

Для форматирования, генерации изображения и компиляции установлено ограничение по времени в 10 секунд,
для выполнения кода~--- 30 секунд.

\subsection{Внутреннее представление DSL}
\label{dsl-classes}

После выполнения кода DSL создается древовидная структура, описывающая код.
Каждая вершина структуры является экземпляром класса, соответствующего той или иной синтаксической
конструкции (см \ref{syntax-items}). Также некоторые вершины могут быть экземплярами
служебных классов и использоваться для преобразований шаблона в процессе генерации программы.
Примером такого класса и соответствующего ему DSL может являться \texttt{RepeatScopeTemplate} и
соответствующая ему функция \texttt{repeat}, которая позволяет повторить несколько раз генерацию
некоторого куска шаблона. К примеру шаблон
\begin{minted}[]{kotlin}
    repeat(2) { i ->
        addFuncCall("print", i)
    }
\end{minted}

сгенерирует следующий код (для языка Python):

\begin{minted}[]{python}
    print(0)
    print(1)    
\end{minted}

\subsection{API}
Для взаимодействия с сервером используется REST API.

Для генерации и получения изображений и текстов программ доступны следующие методы API:

\begin{itemize}
    \item \texttt{/get\_source}~--- получить отформатированный текст программы.
          Возвращает текст программы при наличии задачи (см. передаваемые параметры).
    \item \texttt{/get\_image}~--- получить изображение программы. Возвращает \texttt{html} с изображением.
    \item \texttt{/get\_image\_bytes\_png}~--- получить сырые байты изображения. Используется в ссылке
          в \texttt{html} с изображением.
\end{itemize}

Вышеописанные методы принимают следующие параметры в строке URL:
\begin{itemize}
    \item \texttt{task}~--- имя задачи (шаблона), из которого будет сгенерирована программа.
    \item \texttt{seed}~--- значение, используемое как зерно рандомизации при генерации программы.
    \item Прочие параметры передаются как атрибуты генерации программы из шаблона.
\end{itemize}

Для проверки вывода программы используется метод \texttt{/check\_answer}, который принимает те же
параметры, что и методы получения изображения и текста, а в теле запроса~--- ответ студента.
В процессе выполнения ответ студента построчно сравнивается с выводом программы и возвращается процент
совпавших строк.

Для управления (добавления и удаления) шаблонами программ доступны следующие методы:
\begin{itemize}
    \item \texttt{/add\_task}~--- добавить задачу. Принимает в теле запроса параметры, описанные в
          \ref{template-db}: имя задачи, тэг и шаблон в формате JSON. Данный метод просто сохраняет
          переданные данные в базу
    \item \texttt{/delete\_task}~--- удалить задачу. Принимает в теле запроса структуру, содержащую
          имя удаляемой задачи, в формате JSON. Данный метод удаляет задачу (шаблон) из базы.

\end{itemize}
% (\textit{TODO: примеры кода})
