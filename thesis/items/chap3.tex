\section{Реализация инструмента генерации программ}

\subsection{Используемые технологии}
% * <Mark Zaslavskiy> 18:02:35 06 May 2022 UTC+0300:
% Сначала описывается архитектура, а затем детали реализации

Для разработки инструмента был выбран язык программирования Kotlin \cite{kotlin}. Это язык
программирования, разработанный компанией JetBrains в 2010 году. Основными
преимуществами этого языка программирования являются:
\begin{itemize}
    \item Кроссплатформенность
    \item Возможность интеграции с различными популярными системами сборки
          (Maven, Gradle, etc.)
    \item Возможность интеграции с java без переписывания имеющегося кода
\end{itemize}

Для автоматизации развертывания системы генерации, поддерживания окружения для хранилища и
системы проверки используются технологии Docker \cite{docker} и docker-compose \cite{docker-compose}.
Благодаря Docker можно создать изолированное окружение (контейнер) для компонента, а docker-compose
позволяет объединять контейнеры в единую локальную сеть.
(\textit{TODO: описать для каких компонентов сделаны контейнеры})

Для компонента сервера было принято решение использовать библиотеку Ktor \cite{ktor}, так как
она позволяет быстро создать веб-сервер с нужным функционалом и имеет простой и элегантный API.
% * <Mark Zaslavskiy> 18:05:18 06 May 2022 UTC+0300:
% мне кажется, что надо найти более измеримые плюсы фреймоврка, нежели быстрота разработки (она больше про программиста) или элегантность :)
% ^ <Mark Zaslavskiy> 18:07:51 06 May 2022 UTC+0300:
% также по тексту - меняйте функционал -> функциональность

(\textit{TODO: описание методов API})

(\textit{TODO: примеры кода})

Генерация изображений реализована с помощью библиотеки \texttt{javax.imageio}
\cite{imageio}. По тексту генерируется изображение в формате png.
