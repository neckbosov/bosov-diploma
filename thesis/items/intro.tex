\specialsection{Введение}

В настоящее время знание языка программирования является необходимым для специалиста в отрасли 
информационных технологий, а обучение им - крайне востребованным. 
На сегодняшний день программы по обучению языкам программирования есть не только в университетах,
но и на различных образовательных платформах в интернете.
В связи с ростом числа учащихся подобных курсов и ослабления контакта между студентом и
преподавателем острее встает проблема создания учебных материалов, в частности практических
заданий. Требуется создавать их в большем объеме и в то же время делать их разнообразными
во избежание списывания. Специфически для курсов по изучению языков программирования возникает
необходимость создания множества примеров программ на определенную тему или по конкретному
шаблону. Создание подобных примеров вручную в нескольких вариантах 
(в идеале по отдельности для каждого ученика) затруднительно.
Таким образом, создание удобного программного инструмента,
позволяющего автоматически генерировать примеры кода на различных языках программирования
для учебных задач представляет собой актуальную проблему.
